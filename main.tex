\documentclass{article}
\usepackage[utf8]{inputenc}
\usepackage{amsmath, amssymb}
\usepackage{mathtools}
\usepackage{a4wide}

\title{Optic flow}
\author{Hugo (s214734) \and Mikael H. Hoffmann (s214753) \and (s211469) \and (s194572)}
\date{\today}

\begin{document}

\maketitle

\section{Introduction}
Optical flow is a method to determine object motion in video, based on the relative displacement of pixels
through time in the video. To obtain such a motion prediction, a few strong assumptions are made
about the subject video, including:
\begin{itemize}
    \item There has been only a small displacement through time
    \item There are no changes in the lighting
    \item ???
\end{itemize}
These assumptions are necessary for applying a mathematical model to the problem of optic flow.











\section{Exercise (5)}
Define $A$ as per equation (2) in the handout:
\begin{equation}
    A = \begin{bmatrix}
    \vdots & \vdots \\
    V_{x}(p_i) & V_{y}(p_i) \\
    \vdots & \vdots
    \end{bmatrix}
\end{equation}
Then the normal equation to $Ax = b$ will have the system matrix (where $N$ denotes the considered neighborhood):
\begin{equation}
    A^\intercal A = \begin{bmatrix}
        \sum_{i \in N} V_{x}(p_i)^{2} & \sum_{i \in N} V_{x}(p_i)V_{y}(p_i) \\
        \sum_{i \in N} V_{x}(p_i)V_{y}(p_i) & \sum_{i \in N} V_{y}(p_i)^{2}
    \end{bmatrix}
\end{equation}
Which we could also get by passing a filter of all ones over $N$ in $V_x$ and $V_y$.




\end{document}
