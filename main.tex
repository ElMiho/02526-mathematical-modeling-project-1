\documentclass{article}
\usepackage[utf8]{inputenc}
\usepackage{amsmath, amssymb}
\usepackage{mathtools}
\usepackage{a4wide}
\usepackage{appendix}
\usepackage{listings}

\title{Optic flow}
\author{Hugo (s214734) \and Mikael H. Hoffmann (s214753) \and (s211469) \and (s194572)}
\date{\today}

\begin{document}

\maketitle

\section{Introduction (problem and background)}
Optical flow is a method to determine object motion in video, based on the relative displacement of pixels
through time in the video. To obtain such a motion prediction, a few strong assumptions are made
about the subject video, including:
\begin{itemize}
    \item There has been only a small displacement through time
    \item There are no changes in the lighting
    \item ???
\end{itemize}
These assumptions are necessary for applying a mathematical model to the problem of optic flow.

\section{Data and experiments}




\section{Mathematical model and image processing methodology}
We start by considering the point $\boldsymbol{p} = [p_x, p_y, p_t]$ and we want to find vector $\boldsymbol{u} = [x, y, 1]$ which represents the movement of the above mentioned pixel in \emph{one} time frame, which is given by

\begin{equation}\label{eq:initial-equation}
    \boldsymbol{V(p + u)} - \boldsymbol{V(p)} = 0.
\end{equation}

Here $\boldsymbol{V}$ represents the pixel value at $\boldsymbol{p}$ - therefore the above is equivalent to brightness constancy constraint. 

If the displacement is small then it is possible to do a first order Taylor expansion

\begin{equation}
    \boldsymbol{V(p + u)} = \boldsymbol{V(p)} + \begin{bmatrix}
        \boldsymbol{V}_x(\boldsymbol{p}) & \boldsymbol{V}_y(\boldsymbol{p}) & \boldsymbol{V}_t(\boldsymbol{p})
    \end{bmatrix} \boldsymbol{u}.
\end{equation}

Inserting this in equation \ref{eq:initial-equation} then

\begin{equation}
    \begin{split}
        \boldsymbol{V(p)} + \begin{bmatrix}
        \boldsymbol{V}_x(\boldsymbol{p}) & \boldsymbol{V}_y(\boldsymbol{p}) & \boldsymbol{V}_t(\boldsymbol{p})
        \end{bmatrix} \boldsymbol{u} - \boldsymbol{V(p)} &= \boldsymbol{V}_x(\boldsymbol{p}) x + \boldsymbol{V}_y(\boldsymbol{p}) y + \boldsymbol{V}_t(\boldsymbol{p}) \cdot 1 \\
        &= 0,
    \end{split}
\end{equation}

and rearranging yields

\begin{equation}
    \boldsymbol{V}_x(\boldsymbol{p}) x + \boldsymbol{V}_y(\boldsymbol{p}) y = -\boldsymbol{V}_t(\boldsymbol{p}).
\end{equation}

\subsection*{Lucas Kanade solution - Christian}

Since there is two unknowns and one equation it is not possible to determine a solution - therefore we assume that all pixels, in a small area around $\boldsymbol{p}$, moves in the same way, and therefore we get the overdetermined system\footnote{This is the Lucas-Kanade solution.}

\begin{equation}
    \begin{bmatrix}
        \vdots & \vdots \\
        \boldsymbol{V}_x(\boldsymbol{p}_i) & \boldsymbol{V}_y(\boldsymbol{p}_i) \\
        \vdots & \vdots
    \end{bmatrix} \begin{bmatrix}
        x \\ y
    \end{bmatrix}
    = - \begin{bmatrix}
        \vdots \\ \boldsymbol{V}_t(\boldsymbol{p}_i) \\ \vdots
    \end{bmatrix}.
\end{equation}

\subsection{Image processing methodology - Christian}

\section{Visualise results - Mikael}

\section{Discuss results - }

\section{Conclude}






\section{Exercise (5)}
Define $A$ as per equation (2) in the handout:
\begin{equation}
    A = \begin{bmatrix}
    \vdots & \vdots \\
    V_{x}(p_i) & V_{y}(p_i) \\
    \vdots & \vdots
    \end{bmatrix}
\end{equation}
Then the normal equation to $Ax = b$ will have the system matrix (where $N$ denotes the considered neighborhood):
\begin{equation}
    A^\intercal A = \begin{bmatrix}
        \sum_{i \in N} V_{x}(p_i)^{2} & \sum_{i \in N} V_{x}(p_i)V_{y}(p_i) \\
        \sum_{i \in N} V_{x}(p_i)V_{y}(p_i) & \sum_{i \in N} V_{y}(p_i)^{2}
    \end{bmatrix}
\end{equation}
Which we could also get by passing a filter of all ones over $N$ in $V_x$ and $V_y$.

\newpage
\appendix
\section{Code}
\subsection{Exercise 1}
\lstinputlisting[language=Python,breaklines=true]{code/exercise_1.py}

\subsection{Exercise 2}
\lstinputlisting[language=Python,breaklines=true]{code/exercise_2.py}

\subsection{Exercise 3}
\lstinputlisting[language=Python,breaklines=true]{code/exercise_3.py}


\end{document}
